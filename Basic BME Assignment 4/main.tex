\documentclass[a4paper,15pt]{article}
\usepackage{graphicx}
 

\usepackage{hyperref}
\hypersetup{colorlinks = true,citecolor = blue, linkcolor = blue, urlcolor = blue}

\setlength{\columnsep}{1.5cm}
\setlength{\columnseprule}{1pt}
 

\usepackage{fancyhdr}

\usepackage{biblatex}
\addbibresource{references.bib}

\lhead{Disruptive innovation in healthcare}
\rhead{CONTENTS}
\renewcommand{\headrulewidth}{0.5pt}
\rfoot{Page \thepage}
\fancyfoot[C,C]{Harshit Dhaduk}
\fancyfoot[L,L]{17-Feb-2022}
\renewcommand{\footrulewidth}{0.3pt} 


\begin{document}

\pagestyle{fancy}
\tableofcontents
\empty
\section{What is disruptive innovation?}
Disruptive innovation begins with items that establish low-end or 
new-market footholds and then persistently climb upmarket, eventually
eliminating established competitors.To put it another way, disruption
begins when a company recognises and meets an unmet (and sometimes 
latent) demand.
\section{What leads to disruptive innovation?}
The disruption of industries and marketplaces has been aided by 
technological advancements and information digitalization. It would 
be a mistake, though, to blame disruption only on technology. 
Disruptive innovation is fueled by a number of additional reasons, 
regardless of industry. One or more of the following dynamics are 
common in industries ripe for disruption.
\subsection{Complex experiences}
Few sectors are as complicated as healthcare. "Healthcare is a 
complex adaptive system, meaning that the system's performance and 
behaviour change over time and cannot be completely understood by 
simply knowing about the individual components," says Jeffrey 
Braithwaite, professor of health systems research and president of 
the International Society for Quality in Health Care. No other 
industry or sector compares in terms of scope and diversity—complex 
funding structures, various moving elements, complex customers with 
diverse demands, and a plethora of options and treatments for every 
given person's needs."
\subsection{Consumer confusion}
The insurance industry has a long history that can be traced all the 
way back to ancient Babylon and is documented in the Code of 
Hammurabi.Despite the social benefits of risk management, 
participants and regulators have frequently sought for more openness 
in product definition, pricing, and claims processing. Insurance 
products are typically seen as complex and difficult to understand, 
with contracts full of legalese and jargon. The insurance sector 
appears to be more focused on the business side of things than on 
client demands at times.
\subsection{Redundant intermediaries}
Most retail travel agencies have succumbed to disruptive innovation, 
which was formerly a mainstay of strip mall stores. The travel agent 
used to serve as an intermediary, scouting costs and planning 
itineraries for travellers. Travel agents have become obsolete as a 
result of extensive internet access and sites like Expedia, 
Travelocity, and Orbitz. Consumers can compare flight fares and hotel
accommodations with just a few clicks, allowing them to book with 
simplicity.
\subsection{Limited access}
There are plenty of financial management organisations to choose from
throughout the world. Many financial advisors may prefer to work with
high-net-worth customers who require a variety of services and are 
ready to pay the fees associated with high-touch service. As a result
of this strategy, a substantial number of younger and/or less 
affluent potential investors are asking for assistance in managing 
their finances in a simple, convenient, and low-cost 
manner.Robo-advisors are stepping in to fill the void.
\section{Who benefits from disruptive models?}
Disruptive innovation is clearly beneficial to consumers. Disruptive
products suit customer demands in a market that may have previously 
been unable to meet them due to their ease of access, convenience, 
and cheaper costs.


Although it may seem counterintuitive, it is possible to argue that 
disruption benefits all industry participants. While established 
organisations may face some disruption, disruption forces them to 
reconsider their business models and refocus on their core 
operations. They now have new chances to improve existing products, 
invest in their personnel, and ensure that they are providing 
quality service to their customers. It's possible that wealth 
management may stick around for a long time.

\section{The result of healthcare disruption}
While previous performance is no guarantee of future results,
it's a near-certainty that the disruptive innovation process will 
enhance healthcare, just as it has in so many other industries. 

Health and economic benefits will spread to stakeholders and beyond 
as innovative products fulfil customer demand.

This isn't to say that present industry participants aren't 
important. Hospitals, insurers, providers, and others in the 
healthcare ecosystem can benefit from disruptive innovation, even if
they did not initiate it. Established firms are starting to take 
ownership of innovation, creating leadership roles in some cases to 
help improve their product offerings.
\end{document}
