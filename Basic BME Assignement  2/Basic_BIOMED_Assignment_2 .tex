\documentclass{article}
\usepackage[utf8]{inputenc}


\usepackage{hyperref}  %allows for clickable reference
\hypersetup{colorlinks = true,citecolor = blue, linkcolor = blue, urlcolor = blue}
\usepackage{subcaption}

\usepackage{amsmath}
\usepackage{pifont}
\usepackage{wrapfig}
\usepackage{multicol}
\setlength{\columnsep}{1.5cm}
\setlength{\columnseprule}{1pt}
 
\usepackage{tikz}
\usetikzlibrary{calc}


\usepackage{fancyhdr}

\title{Assignment 2 - Two page write-up on "Evolution of Modern Health Care System"}
\date{February 2, 2022}
\begin{document}
\maketitle
\section{Overview}
 In recent years, Health care and health systems all over the world are undergoing intensive reforms. Internationally, the existing institutions for multilateral cooperation are facing unprecedented challenges. Many institutions are finding it increasingly difficult to fulfill their mandates. There are inefficient overlapping efforts among various multilateral organizations, but paradoxically, there are responsibility voids in executing some key functions. At the same time, other players, such as non-governmental organizations and transnational corporations, are gaining prominence.
 
 Modern healthcare has evolved and become more focused on prevention. Pre-ventative efforts are in place to reduce and eradicate disease, support overall physical and mental health, and educate patients and families to promote safety.Due to this emphasis on prevention as well as our obsession with technology and convenience, it only makes sense that trends and new technology in healthcare focus on providing patients with more access to healthcare so that they can prevent issues before they turn into huge headaches.
 \section{History}
 During the past 150 years, two factors have shaped the modern healthcare system: first, the growth of scientific knowledge about sources and means of controlling disease; second, the growth of public acceptance of disease control as both a possibility and a public responsibility. In earlier centuries, when little was known about the causes of disease, society tended to regard illness with a degree of resignation, and few public actions were taken. As understanding of sources of contagion and means of controlling disease became more refined, more effective interventions against health threats were developed. Public organizations and agencies were formed to employ newly discovered interventions against health threats. As scientific knowledge grew, public authorities expanded to take on new tasks, including sanitation, immunization, regulation, health education, and personal health care.
 \section{Achievements}
 In the past 3 decades, world has witnessed many transitions in healthcare. The major achievements of the last three decades include:
 \begin{itemize}
 
\item Upgrade of health infrastructure
\item More in-depth and reliable knowledge of health systems, health indicators and their challenges
\item Improvement of patient safety and quality in hospital care
\item Use of information technology to promote health care delivery, harnessed by both patients and providers (telemedicine, electronic health records, digital/distance learning, mHealth)
\item Health indicators (IMR, MMR) have been generally improved along with life expectancy
\item New discoveries in health and hospital care such as breakthroughs in human genome and stem cell researches, new and advanced drug therapies extended survival in HIV and cancer patients
\item Minimally invasive and robotic techniques revolutionized surgery

\end{itemize}
\section{Challenges}
Over the last 30 years, health systems around the world have faced multiple challenges. Some of them are listed below:
\begin{itemize}
\item More focus on primary health care and less on secondary tertiary care levels
\item Urbanization and changes in behaviors and diet contributed to an increased prevalence of chronic diseases such as hypertension, coronary artery disease, diabetes and cancer
\item Many less economically developed countries were faced with double burden, which was persistently high rates of infectious diseases combined with rapidly rising rates of chronic diseases
\item Due to a large number of conflicts in different parts of the world over the last 30 years, implementing and sustaining health programs were serious challenges for the health community 
\item  Traumatic injuries, violence and road traffic accidents further strained health systems, resulting in a triple burden, high rates of trauma, infectious and chronic diseases
\item  People have experienced epidemiologic transitions at different rates, which triggered needs for new health services and delivery systems
\end{itemize}




























\end{document}
