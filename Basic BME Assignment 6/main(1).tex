\documentclass[15pt]{article}
\usepackage{graphicx}
\graphicspath{{Images/}}

\usepackage{hyperref}
\hypersetup{colorlinks = true,citecolor = blue, linkcolor = blue, urlcolor = blue}

\setlength{\columnsep}{1.5cm}
\setlength{\columnseprule}{1pt}
 

\usepackage{fancyhdr}

\usepackage{biblatex}
\addbibresource{references.bib}

\lhead{Solutions provided by Biomedical Engineers in this pandemic}
\rhead{CONTENTS}

\renewcommand{\headrulewidth}{0.5pt}
\rfoot{Page \thepage}
\fancyfoot[C,C]{Harshit Dhaduk}
\fancyfoot[L,L]{28-Feb-2022}
\renewcommand{\footrulewidth}{0.3pt} 


\begin{document}
\pagestyle{fancy}
\tableofcontents
\empty

\title{Solutions provided by Biomedical Engineers in this COVID 19 pandemic}

\section{Solving Oxygen shortage}
The delivery of additional oxygen using a nasal cannula or a more 
intrusive face mask is usually the primary form of treatment for 
mild respiratory insufficiency. The oxygen is usually delivered in 
cylinders, which are either tiny for transportation or big for fixed
patients and longer-term supplies.

Although oxygen concentrators are an appealing option to tanks, they
are rarely used in hospital settings for caring for COVID-19 
patients. Oxygen concentrators take oxygen from the air and deliver 
it to the patient on demand. Concentrators are available in a 
variety of sizes, ranging from a small portable shoulder bag to 
larger fixed units for patients who require oxygen 24 hours a day.


\section{dealing with vantilator challenge in UK}
In March 2020, pandemic modelling indicated that the UK would run out of ventilators due to an increase in the number of patients in 
critical care units (ICUs) who needed them. The government 
established the Ventilator Challenge, a call to industry and academia across the country to help alleviate the potential gap.

Many consortia were formed to develop new CPAPs that mostly featured
new ventilator designs or to find ways to scale up production 
capacity of existing ventilator types that were already manufactured
and certified in the UK.

Finally, the demand was lower than expected, the manufacturing 
scale-up efforts were adequate, and none of the new designs were 
subjected to the emergency approval process. This is still a 
fantastic example of what the country's engineering community can do
when faced with a difficult task.


\section{Patient monitoring}
The monitoring equipment, which keeps track of some of the patient's
vitals, especially when they are ventilated and sedated, but also 
during their recovery phase to ensure the ventilation regime is 
optimised for their condition, is an important part of the ICU 
equipment. Ventilators already have their own set of patient 
parameters, but patient monitors are usually distinct devices because they are still relevant when the patient can breathe on their own.

The amount of oxygen in a patient's bloodstream (SpO2), which is 
evaluated by pulse oximetry, which uses optics within a finger clamp, is one of the most important metrics for COVID-19 patients. Pulse oximetry is often used for the duration of a patient's stay in the  intensive care unit.

Modern patient monitors provide a plethora of additional patient 
parameters, all the way down to breathing waveforms, allowing 
doctors to fine-tune their patient treatment.


\section{Continuous Positive Airway Pressure (CPAP)}
Continuous Positive Airway Pressure (CPAP) is the next step in 
treating COVID-19 patients. CPAP was originally designed to avoid 
airways collapse in sleep apnoea patients, but it has been 
demonstrated to be beneficial to COVID patients if used early enough
in the disease's course.

A well-fitting face mask is an important part of a CPAP machine, but
it may be rather bothersome. Because CPAP effectively resists some 
resistance to expiration, it is only suited for patients who are 
capable of some breathing strength. There are variants that adapt the level of pressure automatically to the patient's breathing 
characteristics (APAP) or have distinct levels of pressure for 
inspiration and expiration (BiPAP). CPAP normally provides the 
patient with (filtered) air, but most masks feature a port for adding oxygen to the mix.

\section{COVID Nudge test kit}
Aside from the Ventilator Challenge, the pandemic sparked a slew of 
new ideas from engineers across the country. This section only 
mentions a few of the advances that the authors are aware of, and it
is not intended to pick them out from the rest of the excellent work
that is being done.

During the pandemic, DNA Nudge bioengineers created the COVID Nudge 
test from the ground up.

In response to the COVID-19 outbreak, a team at Imperial College in 
London created JAMVENT, a low-cost emergency ventilator. Its design 
is simple, yet it is capable of performing all of the activities 
required of an ICU ventilator for COVID-19 patients.



\end{document}



